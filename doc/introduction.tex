\section{Introduction}\label{sec:introduction}

Tiny Grace is a tiny subset of the Grace programming language, in the spirit of
Featherweight Java \cite{fj}.  It currently aims to resolve the following
issues:

\begin{itemize}

\item Simple reduction semantics

\item Static typing

\end{itemize}

\noindent Future versions will aim to extend the formalism to address:

\begin{itemize}

\item Safe casting with match-case

\item Type aliases for recursive types

\item Outer semantics and typing

\item Inheritance semantics and typing

\item Dialect semantics

\item Statements and early returns

\item Block semantics and typing

\item Dynamic typing and runtime semantics

\end{itemize}

\noindent Tiny Grace intentionally avoids the following issues:

\begin{itemize}

\item Concrete syntax and sugar, such as mixfix declarations and classes

\item Variables and the heap

\item Generic and inferred types

\item Annotations and reflection

\end{itemize}

\noindent The current version of Tiny Grace is much simpler than Featherweight
Java in most aspects, as it does not feature constructors, fields, or casts, and
does not need to maintain a class table.  Instead of these features, it must
implement scope, as objects can be nested.  This requires a more complicated
substitution algorithm which can substitute both variables and \self, but with
only variables crossing object boundaries.  As the language does not feature
casting, type soundness guarantees program completion, whereas FJ may halt on
invalid casts.  Tiny Grace is strictly less powerful than FJ, as lost type
information cannot be recovered, and recursive types cannot be expressed.

