\section{Properties}

\begin{theorem}[Progress]\label{th:progress}
  Given a well-formed program $(\*\M*, e)$, any expression $e^\prime$ in the
  relation $e \stot e^\prime$ is either an object literal $\O$ or can be reduced
  by an application of $\sto$.

  \begin{lemma}[Substitution Completeness]\label{lem:sub-complete}
    Substitution on a well-typed term always completes.

    \begin{proof}
      A well-typed program must have a finite size, which means every branch
      must bottom out on either a variable, \self, or an empty object literal,
      all of which terminate substitution.  The remaining cases are trivially
      recursive.
    \end{proof}
  \end{lemma}

  \begin{proof}
    A well-typed program has no free variables, so $\xs$ is never encountered in
    reduction.  Inductive steps for method requests $e.m(\*e_i*)$ and
    $\O.m(\*e_i*)$ are trivial, so we need only show that a method request
    $\O.m(\*\O_i*)$ can always proceed and will always complete.
    Rule~\ref{eq:type-obj} guarantees that the requested method always exists in
    the receiver, which is the only requirement for $\sto$ to be applied to this
    form of method request.  The relation is satisfied as long as the
    substitution operation produces an outcome, so by
    Lemma~\ref{lem:sub-complete}, this single step of $\sto$ is guaranteed to
    complete and produce a new expression $e^\prime$.
  \end{proof}
\end{theorem}

\begin{theorem}[Preservation]\label{th:preservation}
  If $\gctx{e}{\tau}$ and $e \sto e^\prime$, then $\gctx{e^\prime}{\tau^\prime}$
  for some $\tau^\prime \sub \tau$.

  \begin{lemma}[Subtyping Reflexivity]\label{lem:sub-reflexive}
    For any type $\tau$, $\tau \sub \tau$.

    \begin{proof}
      By induction on Rule~\ref{eq:type-sub}.  $\type{} \sub \type{}$ is a
      derived axiom.  For every method $\M$ in $\tau$, if $\tau \sub \tau \vdash
      \*\upsilon_{ik} \sub \upsilon_{ik}*, \*\sigma_i \sub \sigma_i*$ then $\tau
      \sub \tau$.  As types cannot be infinite in size, this inductive
      application is guaranteed to complete at every branch on $\tau = \type{}$.
    \end{proof}
  \end{lemma}

  \begin{lemma}[Subtyping Transitivity]\label{lem:sub-transitive}
    For any types $\tau$, $\tau^\prime$, and $\tau^{\prime\prime}$, if $\tau
    \sub \tau^\prime$ and $\tau^\prime \sub \tau^{\prime\prime}$ then $\tau \sub
    \tau^{\prime\prime}$.

    \begin{proof}
      By induction on Rule~\ref{eq:type-sub}.  For every type $\sigma$, $\sigma
      \sub \type$ is a derived axiom.  For every method
      $\mh{m_i}{x_{ik}^{\prime\prime}}{\upsilon_{ik}^{\prime\prime}}
      {\sigma_i^{\prime\prime}}$ in $\tau^{\prime\prime}$, there exists a method
      $\mh{m_i}{x_{ik}^\prime}{\upsilon_{ik}^\prime}{\sigma_i^\prime}$ in
      $\tau^\prime$ where $\*\upsilon_{ik}^{\prime\prime} \sub
      \upsilon_{ik}^\prime*$ and $\*\sigma_i^\prime \sub
      \sigma_i^{\prime\prime}*$, and therefore a method
      $\mh{m_i}{x_{ik}}{\upsilon_{ik}}{\sigma_i}$ where $\*\upsilon_{ik}^\prime
      \sub \upsilon_{ik}*$ and $\*\sigma_i \sub \sigma_i^\prime*$, so if
      $\*\upsilon_{ik}^{\prime\prime} \sub \upsilon_{ik}*$ and $\*\sigma_i \sub
      \sigma_i^{\prime\prime}*$ then $\tau \sub \tau^{\prime\prime}$.  As types
      cannot be infinite in size, this inductive application is guaranteed to
      complete at every branch on $\tau^{\prime\prime} = \type{}$.
    \end{proof}
  \end{lemma}

  \begin{lemma}[Substitution Preservation]\label{lem:preservation}
    If $\ctx{\G, \xs : \upsilon}{e}{\tau}$, and
    $\gctx{e^\prime}{\upsilon^\prime}$ where $\upsilon^\prime \sub \upsilon$,
    then $\gctx{[e^\prime/\xs]e}{\tau^\prime}$ for some $\tau^\prime \sub \tau$.

    \begin{proof}
      By induction on the derivation of $\ctx{\G, \xs : \upsilon}{e}{\tau}$.

      \begin{match}
        \case{\ref{eq:sub-var}}{e = \xs}
        $\ctx{\G, \xs : \upsilon}{\xs}{\upsilon}$, so $\tau = \upsilon$.
        $[e^\prime/\xs]\xs \bto e^\prime$ and $\gctx{e^\prime}{\upsilon^\prime}$,
        so $\tau^\prime = \upsilon^\prime$.  As $\upsilon^\prime \sub \upsilon$,
        $\tau^\prime \sub \tau$.

        \case{\ref{eq:sub-none}}{e = y~\mathbf{where}~y \neq \xs}
        $[e^\prime/\xs]y \bto y$ and $\gctx{y}{\tau}$, so $\tau^\prime = \tau$.
        By Lemma~\ref{lem:sub-reflexive}, $\tau^\prime \sub \tau$.

        \case{\ref{eq:sub-req}}{e = e_0.m(\*e_i*)}
        $\Gamma \vdash e_0 : \tau_0, \*e_i : \sigma_i*$.  By induction,
        $[e^\prime/\xs]e_0 \bto e^\prime_0$ and $\*[e^\prime/\xs]e_i \bto
        e^\prime_i*$ with $\Gamma, \xs : \upsilon \vdash e^\prime_0 :
        \tau_0^\prime, \*e^\prime_i : \sigma_i^\prime*$ where $\tau_0^\prime \sub
        \tau_0$ and $\*\sigma_i^\prime \sub \sigma_i*$.  By
        Rule~\ref{eq:type-req}, there exists a method with head $m(\*x_i :
        \upsilon_i*) \to \tau$ in $\tau_0$ where $\*\sigma_i \sub \upsilon_i*$,
        and as $\tau_0^\prime \sub \tau_0$, by Rule~\ref{eq:type-sub} there exists
        a method with head $m(\*x_i : \upsilon_i^\prime*) \to \sigma$ in
        $\tau_0^\prime$ where $\upsilon_i \sub \upsilon_i^\prime$ and
        $\sigma \sub \tau$.  By Lemma~\ref{lem:sub-transitive}, $\*\sigma_i
        \sub \upsilon_i^\prime*$, so $\ctx{\Gamma, \xs :
        \upsilon}{e_0^\prime.m(\*e_i^\prime*)}{\sigma}$.  This means that $\sigma
        = \tau^\prime$, so $\tau^\prime \sub \tau$.

        \case{\ref{eq:sub-obj}}{e = \object{\overline{\method{m}{\*x_{ik} :
          \upsilon_{ik}*}{\sigma_i}{e_i}}}, \xs = x}
        $x \notin \*x_{ik}*$, so by induction $\*[e^\prime/x]e_i \bto e_i^\prime*$
        with $\*\ctx{\Gamma, x : \upsilon}{e_i^\prime}{\sigma_i^\prime}*$ where
        $\*\sigma_i^\prime \sub \sigma_i*$.  This means the object remains
        well-typed, with no change in the static type of the object, so
        $\tau^\prime = \tau$.  By Lemma~\ref{lem:sub-reflexive}, $\tau^\prime
        \sub \tau$.

        \case{\ref{eq:sub-self}}{e = \object{\*\M_i*}, \xs = \self}
        $[e^\prime/\self]\object{\*\M_i*} \bto \object{\*\M_i*}$ and
        $\gctx{\object{\*\M_i*}}{\tau}$, so $\tau^\prime = \tau$.  By
        Lemma~\ref{lem:sub-reflexive}, $\tau^\prime \sub \tau$.
      \end{match}
    \end{proof}
  \end{lemma}

  \begin{proof}
    By induction on a derivation of $e \sto e^\prime$.  The inductive steps for
    \textsc{R-Rec} and \textsc{R-Arg} are trivial, so only \textsc{R-Req}
    remains.
%
    \begin{align*}
      &e = \object{\*\M_i*}.m(\*\O_i*) &
      \method{m}{\*x_i : \upsilon_i*}{\tau}{e_m} \in \*\M_i* \\
      &e^\prime = [\*\O_i/x_i*, \object{\*\M_i*}/\self]e_m &
    \end{align*}

    \noindent By Rules~\ref{eq:type-req} and~\ref{eq:type-obj}, for some types
    $\*\upsilon^\prime_i*$ we have:
%
    \begin{displaymath}
      \G \vdash \object{\*\methh{\H_j}{e_j}*} : \type{\*\H_j*} \quad\quad
      m(\*x_i : \upsilon_i*) \to \sigma \in \*\H_j* \quad\quad
      \G \vdash \*\O_i : \upsilon^\prime_i* \quad\quad
      \*\upsilon^\prime_i \sub \upsilon_i*
    \end{displaymath}

    \noindent By Rule~\ref{eq:type-meth}, $\ctx{\G,\*x_i :
    \upsilon_i*}{e_m}{\sigma}$ where $\sigma \sub \tau$. By
    Lemma~\ref{lem:preservation}, $\G \vdash [\*\O_i/x_i,\O/\self*]e_m :
    \sigma^\prime$ where $\sigma^\prime \sub \sigma$.  As $\sigma^\prime =
    \tau^\prime$, by Lemma~\ref{lem:sub-transitive}, $\tau^\prime \sub \sigma$.
  \end{proof}

\end{theorem}

\begin{theorem}[Type Soundness]\label{th:type-soundness}
  If $\gctx{e}{\tau}$, then $e \stot \O$ with $\gctx{e}{\tau^\prime}$ where
  $\tau^\prime \sub \tau$.

  \begin{proof}
    Immediate from Theorems~\ref{th:progress} and~\ref{th:preservation}.
  \end{proof}
\end{theorem}

\noindent This proves that for a Tiny Grace program $(\*\M*, e)$, if
$\pjdg{\*\M*}{e}$, then the program will always successfully run to the
production of an object literal.

