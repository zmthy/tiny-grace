\section{Properties}

\begin{theorem}[Progress]
\label{th:progress}

Every well-typed expression is either an object literal or can be reduced by an
application of $\rto$.

\begin{lemma}[Substitution Completeness]
\label{lem:sub-complete}

Substitution on a well-typed term always completes.

\begin{lproof}
A well-typed program must have a finite size, which means every branch must
bottom out on either a variable, \self, or an empty object literal, all of which
terminate substitution. \qed
\end{lproof}

\end{lemma}

\begin{tproof}

$\xs$ is never well-typed in an empty context.  An inductive step for method
requests is trivial, so we need only show that a method request whose receiver
and arguments are all object literals can always proceed and will always
complete.  Rule~\ref{eq:type-obj} guarantees that the requested method always
exists in the receiver, which is the only requirement for $\rto$ to be applied
to this form of method request.  Due to Lemma~\ref{lem:sub-complete}, this
single step of $\rto$ is guaranteed to complete and produce a new expression
$e^\prime$. \qed

\end{tproof}

\end{theorem}

\begin{theorem}[Preservation]
\label{th:preservation}

If $\gctx{e}{\tau}$ and $e \rto e^\prime$, then $\gctx{e^\prime}{\tau^\prime}$
for some $\tau^\prime \sub \tau$.

\begin{lemma}
\label{lem:submethod}

If $\H \in \*\H_i*$, then $\H \in \*\H^\prime_i*$ for all
$\type{\*\H^\prime_i*} \sub \type{\*\H_i*}$.

\begin{lproof}
Follows from Rule~\ref{eq:type-sub}. \qed
\end{lproof}

\end{lemma}

\begin{lemma}[Substitution Preservation]
\label{lem:preservation}

If $\ctx{\G, \xs : \upsilon}{e}{\tau}$, and $\gctx{e^\prime}{\upsilon^\prime}$
where $\upsilon^\prime \sub \upsilon$, then
$\gctx{[e^\prime/\xs]e}{\tau^\prime}$ for some $\tau^\prime \sub \tau$.

\begin{lproof}

By induction on the derivation of $\ctx{\G, \xs : \upsilon}{e}{\tau}$.

\begin{lcase}[S-Var: $e = \xs$]
$\ctx{\G, e : \upsilon}{e}{\upsilon}$, so $\tau = \upsilon$.
$[e^\prime/\xs]e \sto e^\prime$ and $\gctx{e^\prime}{\upsilon^\prime}$, so
$\tau^\prime = \upsilon^\prime$.  As
$\upsilon^\prime \sub \upsilon$, $ \tau^\prime \sub \tau$.
\end{lcase}

\begin{lcase}[S-None: $e = y~\mathbf{where}~y \neq \xs$]
$[e^\prime/\xs]e \sto e$ and $\gctx{e}{\tau}$, so
$\tau^\prime = \tau$.  As $\sub$ is reflexive, $\tau^\prime \sub \tau$.
\end{lcase}

\begin{lcase}[S-Req: $e = e_0.m(\*e_i*)$]
$e_0 \sto e^\prime_0$ and $\*e_i \sto e^\prime_i*$ where
$\Gamma, \xs : \upsilon \vdash e^\prime_0 : \sigma, \*e^\prime_i : \upsilon^\prime*$.
\end{lcase}

\begin{lcase}[S-Obj/S-Meth: $e = \object{\*\M_i*}, \xs = x$]
Trivial inductive step.
\end{lcase}

\begin{lcase}[S-Self: $e = \object{\*\M_i*}, \xs = \self$]
$[e^\prime/\self]e \sto e$ and $\gctx{e}{\tau}$, so
$\tau^\prime = \tau$.  As $\sub$ is reflexive, $\tau^\prime \sub \tau$. \qed
\end{lcase}

\end{lproof}

\end{lemma}

\begin{tproof}

By induction on a derivation of $e \rto e^\prime$.  The inductive step for
\textsc{R-Rec} and \textsc{R-Arg} is trivial, so only \textsc{R-Req} remains.
\begin{align*}
&e = \object{\*\M_i*}.m(\*\O_i*) &
\method{m}{\*x_i : \upsilon_i*}{\sigma}{e_m} \in \*\M_i* \\
&e^\prime = [\*\O_i/x_i*, \object{\*\M_i*}/\self]e_m &
\end{align*}

\noindent By Rules~\ref{eq:type-req} and~\ref{eq:type-obj}, for some types
$\*\upsilon^\prime_i*$ we have:
\begin{displaymath}
\G \vdash \object{\*\methh{\H_j}{e_j}*} : \type{\*\H_j*} \quad\quad
m(\*x_i : \upsilon_i*) \to \sigma \in \*\H_j* \quad\quad
\G \vdash \*\O_i : \upsilon^\prime_i* \quad\quad
\*\upsilon^\prime_i \sub \upsilon_i*
\end{displaymath}

\noindent By Lemma~?
$\*x_i : \upsilon_i*, \self : \type{\H_i} \vdash e_m : \tau$ for some type
$\tau$ where $\tau \sub \sigma$.  By Lemma~?,
$\G, \*x_i : \upsilon_i*, \self : \type{\H_i} \vdash e_m : \tau$.  Then, by
Lemma~\ref{lem:preservation},
$\G \vdash [\*\O_i/x_i,\O/\self*]e_m : \tau^\prime$ for some type
$\tau^\prime \sub \tau$.  $\tau^\prime$ is a subtype of $\sigma$ by transitivity
of $\sub$. \qed

\end{tproof}

\end{theorem}

\begin{theorem}[Type Soundness]
\label{th:type-soundness}

If $\varnothing \vdash e : \tau$ and $e \rtot e^\prime$ with $e^\prime$ in
normal form, then $e^\prime$ is an object literal $\O$ with
$\varnothing \vdash \O : \tau^\prime$ where $\tau^\prime \sub \tau$.

\begin{tproof}
Immediate from Theorems~\ref{th:progress} and~\ref{th:preservation}. \qed
\end{tproof}

\end{theorem}

