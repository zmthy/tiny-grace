\section{Properties}

\begin{theorem}[Progress]\label{th:progress}
  Given a well-formed program $(\*\M*,e)$, any expression $e\′$ in the
  relation $e\stot e\′$ is either an object literal $\O$ or can be reduced
  by an application of $\sto$.

  \begin{lemma}[Substitution Completeness]\label{lem:sub-complete}
    Substitution on a well-typed term always completes.

    \begin{proof}
      A well-typed program must have a finite size, which means every branch
      must bottom out on either a variable, \self, or an empty object literal,
      all of which terminate substitution. The remaining cases are trivially
      recursive.
    \end{proof}
  \end{lemma}

  \begin{proof}
    A well-typed program has no free variables, so $\xs$ is never encountered in
    reduction. Inductive steps for method requests $e.m(\*e_a*)$ and
    $\O.m(\*e_a*)$ are trivial, so we need only show that a method request
    $\O.m(\*\O_a*)$ can always proceed and will always complete.
    Rule~\ref{eq:type-obj} guarantees that the requested method always exists in
    the receiver, which is the only requirement for $\sto$ to be applied to this
    form of method request. The relation is satisfied as long as the
    substitution operation produces an outcome, so by
    Lemma~\ref{lem:sub-complete}, this single step of $\sto$ is guaranteed to
    complete and produce a new expression $e\′$.
  \end{proof}
\end{theorem}

\begin{theorem}[Preservation]\label{th:preservation}
  If $\gctx{e}{\tau}$ and $e\sto e\′$, then $\gctx{e\′}{\tau\′}$
  for some $\tau\′\sub\tau$.

  \begin{lemma}[Subtyping Reflexivity]\label{lem:sub-reflexive}
    For any type $\tau$, $\tau \sub \tau$.

    \begin{proof}
      By induction on Rule~\ref{eq:type-sub}. $\type{}\sub\type{}$ is a
      derived axiom. For every method $\meth{m}{x}{\tau_p}{\tau_r}$ in $\tau$,
      if $\cjdg{\*\tau_p\sub\tau_p*,\*\tau_r\sub\tau_r*}{\tau\sub\tau}$, then
      $\tau\sub\tau$. As types cannot be infinite in size, this inductive
      application is guaranteed to complete at every branch on $\tau=\type{}$.
    \end{proof}
  \end{lemma}

  \begin{lemma}[Subtyping Transitivity]\label{lem:sub-transitive}
    \newcommand\ta{\tau^{_a}}
    \newcommand\tb{\tau^{_b}}
    \newcommand\tc{\tau^{_c}}
    \newcommand\xa{x^{_a}}
    \newcommand\xb{x^{_b}}
    \newcommand\xc{x^{_c}}

    For any types $\ta$, $\tb$, and $\tc$, if $\ta\sub\tb$ and $\tb\sub\tc$ then
    $\ta\sub\tc$.

    \begin{proof}
      By induction on Rule~\ref{eq:type-sub}. For every type $\tau$,
      $\tau\sub\type{}$ is a derived axiom. For every method
      $\meth{m}{\xc}{\tc_p}{\tc_r}$ in $\tc$, there exists a method
      $\meth{m}{\xb}{\tb_p}{\tb_r}$ in $\tb$ where $\*\tc_p\sub\tb_p*$ and
      $\tb_r\sub\tc_r$, and therefore a method $\meth{m}{\xa}{\ta_p}{\ta_r}$
      where $\*\tb_p\sub\ta_p*$ and $\ta_r\sub\tb_r$, so if
      $\*\tc_p\sub\tc_p*$ and $\ta_r\sub\tc_r$ then $\ta\sub\tc$. As types
      cannot be infinite in size, this inductive application is guaranteed to
      complete at every branch on $\tc = \type{}$.
    \end{proof}
  \end{lemma}

  \begin{lemma}[Substitution Preservation]\label{lem:preservation}
    If $\ctx{\G,\xs:\tau_x}{e}{\tau}$, and $\gctx{\O}{\tau_o}$ for any $\O$
    where $\tau_o\sub\tau_x$, then $\gctx{[\O/\xs]e}{\tau\′}$ for some
    $\tau\′\sub\tau$.

    \begin{proof}
      By induction on the derivation of $\ctx{\G,\xs:\tau_x}{e}{\tau}$.

      \begin{match}
        \case{\ref{eq:sub-var}}{e=\xs}
        $\ctx{\G,\xs:\tau_x}{\xs}{\tau_x}$, so $\tau=\tau_x$.
        $[\O/\xs]\xs\bto\O$ and $\gctx{\O}{\tau_o}$. $\tau_o\sub\tau_x$, so
        $\tau_o\sub\tau$.

        \case{\ref{eq:sub-none}}{e=y~\mathbf{where}~y\neq\xs}
        $[\O/\xs]y\bto y$ and $\gctx{y}{\tau}$. By
        Lemma~\ref{lem:sub-reflexive}, $\tau\sub\tau$.

        \case{\ref{eq:sub-req}}{e=e_r.m (\*e_a*)}
        $\G\vdash e_r:\tau_r,\*e_a:\tau_a*$. By induction, $[\O/\xs]e_r\bto
        e_r\′$ and $\*[\O/\xs]e_a\bto e_a\′*$ with $\G,\xs:\tau_x\vdash
        e_r\′:\tau_r\′,\*e\′_a:\tau_a\′*$ where $\tau_r\′\sub\tau_r$ and
        $\*\tau_a\′\sub\tau_a*$. By Rule~\ref{eq:type-req}, there exists a
        method with head $m(\*x_p:\tau_p*)\to\tau$ in $\tau_r$ where
        $\*\tau_a\sub\tau_p*$. As $\tau_r\′\sub\tau_r$, by
        Rule~\ref{eq:type-sub} and there exists a method with head
        $m(\*x_p\′:\tau_p\′*)\to\tau\′$ in $\tau_r\′$ where
        $\*\tau_p\sub\tau_p\′*$ and $\tau\′\sub\tau$. $\*\tau_a\sub\tau_p*$ and
        $\*\tau_p\sub\tau_p\′*$, so by Lemma~\ref{lem:sub-transitive},
        $\*\tau_a\sub\tau_p\′*$. Then by Rule~\ref{eq:type-req},
        $\gctx{e_r\′.m(e_a\′)}{\tau\′}$.

        \case{\ref{eq:sub-obj}}
          {e=\object{\overline{\method{m}{\*x_p:\tau_p*}{\tau_s}{e_b}}},\xs=x}
        By Rule~\ref{eq:type-req},
        $\overline{\ctx{\G,\*x_p:\tau_p*}{e_b}{\tau_b}}$ where
        $\tau_b\sub\tau_s$. $x\notin\*x_p*$, so by induction $\*[\O/x]e_b\bto
        e_b\′*$ with $\*\ctx{\G,\*x_p:\tau_p*}{e_b\′}{\tau_b\′}*$ where
        $\*\tau_b\′\sub\tau_b*$.  As $\*\tau_b\′\sub\tau_b*$ and
        $\*\tau_b\sub\tau_s*$, by Lemma~\ref{lem:sub-transitive},
        $\*\tau_b\′\sub\tau_s*$. As the types of the methods have not changed,
        $\gctx{\object{\overline{\method{m}{\*x_p:\tau_p*}{\tau_s}{e_b\′}}}}
        {\tau}$, and by Lemma~\ref{lem:sub-reflexive}, $\tau\sub\tau$.

        \case{\ref{eq:sub-self}}{e = \object{\*\M_i*}, \xs= \self}
        $[\O/\self]\object{\*\M*}\bto\object{\*\M*}$ and
        $\gctx{\object{\*\M*}}{\tau}$. By Lemma~\ref{lem:sub-reflexive},
        $\tau\sub\tau$.
      \end{match}
    \end{proof}
  \end{lemma}

  \begin{proof}
    By induction on a derivation of $e\sto e\′$. The inductive steps for
    \textsc{R-Rec} and \textsc{R-Arg} are trivial, so only \textsc{R-Req}
    remains.
%
    \begin{align*}
      &e=\object{\*\M*}.m(\*\O_a*)
      &\method{m}{\*x_p:\tau_p*}{\tau}{e_b}\in\*\M*\\
      &e\′=[\*\O_a/x_p*,\object{\*\M*}/\self]e_b&
    \end{align*}

    \noindent By Rules~\ref{eq:type-req} and~\ref{eq:type-obj}, for some types
    $\*\tau_a*$ we have $\G\vdash\*\O_a:\tau_a*$ and $\*\tau_a\sub\tau_p*$. By
    Rule~\ref{eq:type-meth}, $\ctx{\G,\*x_p:\tau_p*}{e_b}{\tau_b}$ where
    $\tau_b\sub\tau$. By Lemma~\ref{lem:preservation}, $\gctx{e\′}{\tau_b\′}$
    where $\tau_b\′\sub\tau$, and so by Lemma~\ref{lem:sub-transitive},
    $\tau_b\′\sub\tau$.
  \end{proof}
\end{theorem}

\begin{theorem}[Type Soundness]\label{th:type-soundness}
  If $\gctx{e}{\tau}$ and $e\stot e\′$ with $e\′$ in normal form
  (there is no $e^{\prime\prime}$ such that $e\′\sto e^{\prime\prime}$),
  then $e\′$ is an object literal $\O$ with $\gctx{\O}{\tau\′}$ where
  $\tau\′\sub\tau$.

  \begin{proof}
    Immediate from Theorems~\ref{th:progress} and~\ref{th:preservation}.
  \end{proof}
\end{theorem}

\noindent This proves that for any Tiny Grace program $(\*\M*, e)$, if
$\pjdg{\*\M*}{e}$, then the program will always successfully run to the
production of an appropriate object literal or continue indefinitely.

