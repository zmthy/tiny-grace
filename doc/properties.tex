\section{Properties}

\begin{theorem}[Progress]
  \label{th:progress}

  Every well-typed expression is either an object literal or can be reduced by
  an application of $\sto$.

  \begin{lemma}[Substitution Completeness]
    \label{lem:sub-complete}

    Substitution on a well-typed term always completes.

    \begin{proof}
      A well-typed program must have a finite size, which means every branch
      must bottom out on either a variable, \self, or an empty object literal,
      all of which terminate substitution.
    \end{proof}
  \end{lemma}

  \begin{proof}
    $\xs$ is never well-typed in an empty context.  An inductive step for method
    requests is trivial, so we need only show that a method request whose
    receiver and arguments are all object literals can always proceed and will
    always complete.  Rule~\ref{eq:type-obj} guarantees that the requested
    method always exists in the receiver, which is the only requirement for
    $\sto$ to be applied to this form of method request.  Due to
    Lemma~\ref{lem:sub-complete}, this single step of $\sto$ is guaranteed to
    complete and produce a new expression $e^\prime$.
  \end{proof}
\end{theorem}

\begin{theorem}[Preservation]
  \label{th:preservation}

  If $\gctx{e}{\tau}$ and $e \sto e^\prime$, then $\gctx{e^\prime}{\tau^\prime}$
  for some $\tau^\prime \sub \tau$.

  \begin{lemma}
    \label{lem:submethod}

    If $\H \in \*\H_i*$, then $\H \in \*\H^\prime_i*$ for all
    $\type{\*\H^\prime_i*} \sub \type{\*\H_i*}$.

    \begin{proof}
      Follows from Rule~\ref{eq:type-sub}.
    \end{proof}
  \end{lemma}

  \begin{lemma}[Substitution Preservation]
    \label{lem:preservation}

    If $\ctx{\G, \xs : \upsilon}{e}{\tau}$, and
    $\gctx{e^\prime}{\upsilon^\prime}$ where $\upsilon^\prime \sub \upsilon$,
    then $\gctx{[e^\prime/\xs]e}{\tau^\prime}$ for some $\tau^\prime \sub \tau$.

    \begin{proof}
      By induction on the derivation of $\ctx{\G, \xs : \upsilon}{e}{\tau}$.

      \begin{match}
        \case{\ref{eq:sub-var}}{e = \xs}
        $\ctx{\G, e : \upsilon}{e}{\upsilon}$, so $\tau = \upsilon$.
        $[e^\prime/\xs]e \bto e^\prime$ and $\gctx{e^\prime}{\upsilon^\prime}$,
        so $\tau^\prime = \upsilon^\prime$.  As $\upsilon^\prime \sub \upsilon$,
        $\tau^\prime \sub \tau$.

        \case{\ref{eq:sub-none}}{e = y~\mathbf{where}~y \neq \xs}
        $[e^\prime/\xs]e \bto e$ and $\gctx{e}{\tau}$, so $\tau^\prime = \tau$.
        As $\sub$ is reflexive, $\tau^\prime \sub \tau$.

        \case{\ref{eq:sub-req}}{e = e_0.m(\*e_i*)}
        $e_0 \bto e^\prime_0$ and $\*e_i \bto e^\prime_i*$ where $\Gamma, \xs :
        \upsilon \vdash e^\prime_0 : \sigma, \*e^\prime_i : \upsilon^\prime*$.

        \case{\ref{eq:sub-obj}}{e = \object{\*\M_i*}, \xs = x}
        Step into method bodies.  Trivial inductive step, as parameters may not
        shadow variables.

        \case{\ref{eq:sub-self}}{e = \object{\*\M_i*}, \xs = \self}
        $[e^\prime/\self]e \bto e$ and $\gctx{e}{\tau}$, so $\tau^\prime =
        \tau$. As $\sub$ is reflexive, $\tau^\prime \sub \tau$.
      \end{match}
    \end{proof}
  \end{lemma}

  \begin{proof}
    By induction on a derivation of $e \sto e^\prime$.  The inductive steps for
    \textsc{R-Rec} and \textsc{R-Arg} are trivial, so only \textsc{R-Req}
    remains.
%
    \begin{align*}
      &e = \object{\*\M_i*}.m(\*\O_i*) &
      \method{m}{\*x_i : \upsilon_i*}{\sigma}{e_m} \in \*\M_i* \\
      &e^\prime = [\*\O_i/x_i*, \object{\*\M_i*}/\self]e_m &
    \end{align*}

    \noindent By Rules~\ref{eq:type-req} and~\ref{eq:type-obj}, for some types
    $\*\upsilon^\prime_i*$ we have:
%
    \begin{displaymath}
      \G \vdash \object{\*\methh{\H_j}{e_j}*} : \type{\*\H_j*} \quad\quad
      m(\*x_i : \upsilon_i*) \to \sigma \in \*\H_j* \quad\quad
      \G \vdash \*\O_i : \upsilon^\prime_i* \quad\quad
      \*\upsilon^\prime_i \sub \upsilon_i*
    \end{displaymath}

    \noindent By Lemma~? $\*x_i : \upsilon_i*, \self : \type{\H_i} \vdash e_m :
    \tau$ for some type $\tau$ where $\tau \sub \sigma$.  By Lemma~?, $\G, \*x_i
    : \upsilon_i*, \self : \type{\H_i} \vdash e_m : \tau$.  Then, by
    Lemma~\ref{lem:preservation}, $\G \vdash [\*\O_i/x_i,\O/\self*]e_m :
    \tau^\prime$ for some type $\tau^\prime \sub \tau$.  $\tau^\prime$ is a
    subtype of $\sigma$ by transitivity of $\sub$.
  \end{proof}

\end{theorem}

\begin{theorem}[Type Soundness]
  \label{th:type-soundness}

  If $\gctx{e}{\tau}$, then $e \stot \O$ with $\gctx{e}{\tau^\prime}$ where
  $\tau^\prime \sub \tau$.

  \begin{proof}
    Immediate from Theorems~\ref{th:progress} and~\ref{th:preservation}.
  \end{proof}
\end{theorem}

\noindent This proves that for a Tiny Grace program $(\*\M*, e)$, if
$\pjdg{\*\M*}{e}$, then the program will always successfully run to the
production of an object literal.

