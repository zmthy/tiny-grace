\section{Language}
\label{sec:language}

A Tiny Grace program consists of a pair containing an object body and an
expression which will be evaluated in the scope of the body. Unlike standard
Grace, parentheses and type information is always required.  As there is no type
aliasing, type information is always a type literal, so the current language is
fairly verbose.  Tiny Grace programs are usually valid Grace programs, with the
exception that variables may shadow method names.

As an example, we will model a constructor that takes an object, and `wraps' it
in another object, behind the \lstinline{value} method.

\begin{lstlisting}
method wrap(x : {}) -> { value() -> {} } {
  object {
    method value() -> {} { x }
  }
}
\end{lstlisting}

\noindent Due to subtyping, any object may be passed to this method as
\lstinline{x}, as all objects satisfy the empty type.  Any static knowledge of
the methods in \lstinline{x} is lost, and cannot be recovered, so no methods can
ever be requested on the result of the \lstinline{value} method.

The \lstinline{wrap} method is a valid object body by itself, and we can now
evaluate an expression in the scope of this body.  Method requests cannot be
unqualified, and so the method must be requested as \lstinline{self.wrap}.

\begin{lstlisting}
self.wrap(object {
  method id(y : {}) -> {} { y }
})
\end{lstlisting}

\noindent The outcome of requesting \lstinline{value} on this expression will be
the object passed to \lstinline{wrap}.  As mentioned above, the type checker
will not allow a request for the method \lstinline{id} on the result.

\subsection{Syntax}
\label{sec:syntax}

The abstract syntax for Tiny Grace is defined in
Figure~\ref{fig:abstract-syntax}.  The metavariable $\M$ ranges over methods;
$\O$ over object literals; $x$ and $y$ over variable names; $\xs$ over both
variable names and the {\sffamily\bfseries self} keyword; $\tau$, $\sigma$ and
$\upsilon$ over types; $\H$ over method headers, $m$ over method names, and $e$
over expressions.  We write $\*e*$ to indicate a possibly empty sequence of
comma-separated expressions $e_1, \dots, e_n$, as well as for method header
parameters $\*x : \tau*$.  We also write $\*\H*$ and $\*\M*$ to indicate a
possibly empty set of method headers with unique names $\H_1 \dots \H_n$ and
$\M_1 \dots \M_n$ respectively.

As an object body is just a sequence of methods, a Tiny Grace program is
any pair of the form $(\*\M*, e)$.

\begin{figure}[h]
  \centering

  \grammarindent3.4em
  \renewcommand{\grammarlabel}[2]{$#1$\hfill#2}
  \renewcommand{\syntleft}{\itshape}
  \renewcommand{\syntright}{}
  \renewcommand{\ulitleft}{\sffamily\bfseries}
  \renewcommand{\litleft}{\sffamily}
  \renewcommand{\litright}{}

  \vspace{1em}
  \begin{minipage}{11.7em}
    \begin{grammar}
      <\M> ::= "method" $\H$ \bo~$e$ \bc

      <\O> ::= "object" \bo~$\*\M*$ \bc

      <\H> ::= <m>`('$\*x : \tau*$`)' $\to$ $\tau$

      <\tau> ::= \bo~$\*\H*$ \bc

      <e> ::= <x> | "self" | <e>`.'<m>($\*e*$) | $\O$
    \end{grammar}
  \end{minipage}

  \caption{Tiny Grace abstract syntax}
  \label{fig:abstract-syntax}
\end{figure}

