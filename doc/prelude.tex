% Document information.

\newcommand{\version}{0.1.1}

\newcommand{\thetitle}{Tiny Grace Formal Specification}
\newcommand{\theauthor}{Timothy Jones}

\usepackage[
  pdftitle={\thetitle},
  pdfauthor={\theauthor},
  colorlinks,
  urlcolor=black,
  linkcolor=black,
  citecolor=black
]{hyperref}


% Math definitions.

\newcommand{\ex}[2]{\exists#1.~#2}
\newcommand{\fa}[2]{\forall#1.~#2}

% Environment definitions.
\renewcommand{\emptyset}{\varnothing}
\newcommand{\G}{\Gamma}
\newcommand{\stjdg}[2]{#1\vdash#2}
\newcommand{\tjdg}[3]{\stjdg{#1}{#2:#3}}
\newcommand{\gtjdg}[2]{\tjdg{\G}{#1}{#2}}
\newcommand{\etjdg}[2]{\tjdg{\emptyset}{#1}{#2}}
\newcommand{\oltjdg}[3]{\stjdg{#1}{\ol{#2:#3}}}
\newcommand{\olgtjdg}[2]{\oltjdg{\G}{#1}{#2}}
\newcommand{\oletjdg}[2]{\oltjdg{\emptyset}{#1}{#2}}
\newcommand{\cjdg}[2]{#1\mid#2}
\newcommand{\sjdg}[2]{#1\sub#2}
\newcommand{\mjdg}[2]{#1\vdash#2~\checkmark}
\newcommand{\gmjdg}[1]{\mjdg{\G}{#1}}
\newcommand{\pjdg}[2]{(#1, #2)~\checkmark}

% Subscript shorthands.
\newcommand{\prmt}{\tau_p}
\newcommand{\rett}{\tau_r}
\newcommand{\rect}{\tau_s}
\newcommand{\prme}{e_p}
\newcommand{\rete}{e_r}
\newcommand{\rece}{e_s}
\newcommand{\recO}{O_s}
\newcommand{\prmO}{O_p}
\newcommand{\prmx}{x_p}

% Multiplicities with overline.
\newcommand{\ol}[1]{\overline{#1}}

% Shorter superscript prime.
\def\′{^\prime}

% Subtyping relation.
\newcommand{\sub}{<\kern-1pt:}

% Braces and parentheses.
\newcommand{\bo}{\text{\sffamily\upshape\{}}
\newcommand{\bc}{\text{\sffamily\upshape\}}}
\newcommand{\po}{\text{\sffamily\upshape(}}
\newcommand{\pc}{\text{\sffamily\upshape)}}

% Tiny Grace keywords and constructs.
\newcommand{\key}[1]{\text{\sffamily\bfseries#1}}
\newcommand{\oc}[1]{\bo~\!#1~\!\bc}
\newcommand{\ocol}[1]{\oc{\ol{#1}}}
\newcommand{\koc}[2]{\key{#1}~\!\oc{#2}}
\newcommand{\object}[1]{\koc{object}{#1}}
\newcommand{\etype}{\key{type}~\!\bo\bc}
\newcommand{\type}[1]{\koc{type}{#1}}
\newcommand{\meth}[2]{\key{method}~#1~\oc{#2}}
\newcommand{\method}[4]{\meth{#1\po#2\pc\to#3}{#4}}
\newcommand{\header}[4]{#1\po\ol{#2:#3}\pc\to#4}
\newcommand{\self}{\text{\sffamily\upshape{self}}}
\newcommand{\param}[2]{\text{\lstinline{#1}}:\type{#2}}
\newcommand{\objectol}[1]{\object{\ol{#1}}}
\newcommand{\typeol}[1]{\type{\ol{#1}}}
\newcommand{\range}[2]{#1\dots#2}
\newcommand{\cons}[1]{~;~\!\ol{#1}}

% Small-step reduction relation.
\newcommand{\ssr}{\longrightarrow}
\newcommand{\ssrt}{\longrightarrow^*}
\newcommand{\sto}[2]{#1\longrightarrow#2}
\newcommand{\stot}[2]{#1\longrightarrow^*#2}

% Big-step reduction relation.
\newcommand{\bto}{\Downarrow}


% Listings settings.

\lstdefinelanguage{TinyGrace}{%
  keywords={object,method,type,outer},
  sensitive=true,
  literate=
    {'}{$^\prime$}1
    {->\ }{$\to$ }1
}

\lstset{%
  basicstyle=\sffamily,
  keywordstyle=\bfseries,
  xleftmargin=1.5em,
  columns=fullflexible,
  language=TinyGrace,
  mathescape
}


% Theorem definitions.

\newcounter{theorem}
\newenvironment{theorem}[2]
  {\let\oldparindent=\parindent\setlength{\parindent}{0em}
    \textsc{Theorem}~\refstepcounter{theorem}\thetheorem~(#1)\label{th:#2}
    \newcommand{\lbl}[1]{\label{th:#2:##1}}
    \newcommand{\trf}[1]{\ref{th:#2:##1}}\vspace{1em}}
  {\setlength{\parindent}{\oldparindent}\vspace{1em}}

\newcounter{lemma}
\newenvironment{lemma}[2]
  {\let\oldparindent=\parindent\setlength{\parindent}{0em}
    \textsc{Lemma}~\refstepcounter{lemma}\thelemma~(#1)\label{lem:#2}
    \newcommand{\lbl}[1]{\label{lem:#2:##1}}
    \newcommand{\trf}[1]{\ref{lem:#2:##1}}\vspace{1em}}
  {\setlength{\parindent}{\oldparindent}\vspace{1em}}

\newenvironment{proof}
  {\vspace{1em}\noindent\textsc{Proof.}\stepcounter{steps}}{}

\newcounter{conds}
\newcounter{cond}[conds]
\renewcommand{\thecond}{\bf\alph{cond}}
\newenvironment{conds}
  {\textit{If:}\stepcounter{conds}
    \renewcommand{\cond}[1]{\par\hspace{2em}\numbox{cond}{\thecond}##1}}{}

\newcommand{\cond}[1]{\begin{conds}\cond{#1}\end{conds}}
\newcommand{\then}{\par\textit{then:}}
\newcommand{\where}{\par\textit{where:}}
\newcommand{\tor}{\par\textit{or:}}
\newcommand{\axiom}[1]{\par\hspace{2em}#1}

\newcommand{\listindent}{\addtolength\leftskip{2em}}

\newcounter{steps}
\newenvironment{case}[1]
  {\vspace{.5em}\textit{Case} #1\vspace{.5em}\stepcounter{steps}\par
    \listindent}{\par}
\newenvironment{analysis}[1]
  {\vspace{.5em}Case analysis on #1:\par
    \renewcommand{\thestep}{\bf\roman{step}}}{}

\newcommand{\stepwidth}{.62\textwidth-\leftskip}
\newcommand{\numwidth}{2.5em}

\newcounter{step}[steps]
\renewcommand{\thestep}{\bf\arabic{step}}

\newcommand{\numbox}[2]{\refstepcounter{#1}\makebox[\numwidth][l]{#2.}}
\newcommand{\stepnum}{\numbox{step}{\thestep}}

\newcommand{\slet}[1]{\par\stepnum\textup{let}~#1}
\newcommand{\step}[2]{\par\stepnum\makebox[\stepwidth][l]{#1}by #2}

\newsavebox{\stepsby}
\newenvironment{steps}[1]
  {\savebox{\stepsby}{#1}
    \renewcommand{\step}[1]{\stepnum\makebox[\stepwidth][l]{##1}}
    \par\begin{math}\left.\hspace{-.1em}\begin{minipage}{\numwidth+\stepwidth}}
  {\end{minipage}\right\}\end{math} by \usebox{\stepsby}\par}

\newenvironment{for}[1]{\vspace{.5em}\textit{for all}~#1\par\listindent}
  {\par\vspace{.5em}}

\newcommand{\na}{case N/A}
\newcommand{\done}[1]{\step{done}{#1}}
\newcommand{\byeq}[1]{\ref{eq:#1}}
\newcommand{\bydef}[1]{def~\ref{eq:#1}}
\newcommand{\byind}{ind hyp}
\newcommand{\bycontra}{contradiction}
\newcommand{\bysubst}{substitution}
\newcommand{\byrewrite}{Barendregt}
\newcommand{\byprem}[1]{premise of~\ref{eq:#1}}
\newcommand{\byprems}[1]{premises of~\ref{eq:#1}}
\newcommand{\bypremss}[2]{premises of~\ref{eq:#1},~\ref{eq:#2}}
\newcommand{\bylem}[1]{lemma~\ref{lem:#1}}
\newcommand{\bylems}[2]{lemmas~\ref{lem:#1},~\ref{lem:#2}}
\newcommand{\byth}[1]{theorem~\ref{th:#1}}
\newcommand{\bythh}[2]{theorems~\ref{th:#1},~\ref{th:#2}}

\newcommand{\indmsg}[1]{induction on the derivation of #1}
\newcommand{\ind}[1]{By straightforward \indmsg{#1}.}
\newcommand{\indana}[1]{By \indmsg{#1}, with a case analysis on the last step:}

