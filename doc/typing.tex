\section{Typing}\label{sec:typing}

A Tiny Grace program $(\ol{\methh{H}{e_b}}, e)$ is well-formed if we can produce
a type $\tau$ for $e$ in the context of its object body. This can be expressed
with the typing rules in Section~\ref{sec:typing-rules} as:

\begin{equation}
  \tag{\textsc{T-Prg}}\label{eq:type-prg}
  \AxiomC{$\tjdg{\varnothing}{\object{\ol{\methh{H}{e_b}}}}{\type{\ol{H}}}$}
  \AxiomC{$\tjdg{\self : \type{\ol{H}}}{e}{\tau}$}
  \BinaryInfC{$\pjdg{\ol{\methh{H}{e_b}}}{e}$}
  \DisplayProof
\end{equation}

\subsection{Subtyping}\label{sec:subtyping}

The rules for subtyping are given in Figure~\ref{fig:subtyping}. The relation is
written $\ssjdg\tau{\tau\′}$, meaning a type $\tau$ is a subtype of type
$\tau\′$ in a constraint set $\Sigma$ if, for every method in $\tau\′$, there is
a method with the same name in $\tau$ with contravariant parameter types and a
covariant return type. Subtyping with an empty assumption set is abbreviated
$\tau\sub\tau\′$. Constraints may be added to the constraint set with
$\Sigma,\tau\sub\tau\′$, which implies the constraint is not already in the set.

\begin{figure}[h]
  \centering

  \begin{equation}
    \tag{\textsc{T-Assum}}\label{eq:type-assum}
    \AxiomC{$\tau\sub\tau\′\in\Sigma$}
    \UnaryInfC{$\ssjdg{\tau}{\tau\′}$}
    \DisplayProof
  \end{equation}
%
  \begin{equation}
    \tag{\textsc{T-Sub}}\label{eq:type-sub}
    \AxiomC{$\mh{m}{x}{\tau_p}{\tau}\subseteq\ol{H}$}
    \AxiomC{$\ol{\tau_p\′\sub\tau_p},\ol{\tau\sub\tau\′}\mid
      \Sigma,\type{\ol{H}}\sub\mset{m}{x\′}{\tau_p\′}{\tau\′}$}
    \BinaryInfC{$\ssjdg{\type{\ol{H}}}{\mset{m}{x\′}{\tau_p\′}{\tau\′}}$}
    \DisplayProof
  \end{equation}

  \caption{Tiny Grace subtyping}\label{fig:subtyping}
\end{figure}

\subsection{Typing Rules}\label{sec:typing-rules}

The typing rules for expressions and method declarations is given in
Figure~\ref{fig:typing}. An environment $\G$ is a finite mapping from variable
names to types, written $\ol{\xs_i : \tau_i}$. The empty environment is written
$\varnothing$. Environments can be concatenated together, typically written
$\G,\ol{\xs_i : \tau_i}$, with variables that exist in both environments bound to
the mapping in the latter environment. The typing judgement for expressions has
the form $\gtjdg{e}{\tau}$, meaning that in the environment $\G$, $e$ has the
type $\tau$. The typing judgement for methods has the form $\gmjdg{M}$, meaning
that in the environment $\G$, the method $M$ is well-formed. The typing
judgement for programs has the form $\pjdg{\ol{M}}{e}$, meaning that the program
$(\ol{M}, e)$ is well-formed.

\begin{figure}[h]
  \centering

  \newcommand{\name}[1]{\tag{\textsc{T-#1}}}

  \begin{equation}
    \name{Var}\label{eq:type-var}
    \AxiomC{$\xs:\tau\in\G$}
    \UnaryInfC{$\gtjdg{\xs}{\tau}$}
    \DisplayProof
  \end{equation}
%
  \begin{equation}
    \name{Req}\label{eq:type-req}
    \AxiomC{$\gtjdg{e}{\type{\ol{H}}}$}
    \AxiomC{$\meth{m}{x}{\tau_p}{\tau}\in\ol{H}$}
    \AxiomC{$\gtjdg{\ol{e_a}{\tau_a}}$}
    \AxiomC{$\ol{\tau_a\sub\tau_p}$}
    \QuaternaryInfC{$\gtjdg{e.m(\ol{e_a})}{\tau}$}
    \DisplayProof
  \end{equation}
%
  \begin{equation}
    \name{Obj}\label{eq:type-obj}
    \AxiomC{$\overline{m_i\notin\ol{H}\setminus H_i}$}
    \AxiomC{$\G,\self:\type{\ol{H}} \vdash
      \ol{\methh{H}{e}~\checkmark}$}
    \BinaryInfC{$\gtjdg{\object{\ol{\methh{H}{e}}}}{\type{\ol{H}}}$}
    \DisplayProof
  \end{equation}
%
  \begin{equation}
    \name{Meth}\label{eq:type-meth}
    \AxiomC{$\ol{x\notin\G}$}
    \AxiomC{$\tjdg{\G,\ol{x:\tau_p}}{e}{\tau}$}
    \AxiomC{$\tau\sub\tau\′$}
    \TrinaryInfC{$\gmjdg{\method{m}{\ol{x:\tau_p}}{\tau\′}{e}}$}
    \DisplayProof
  \end{equation}

  \caption{Tiny Grace expression and method typing rules}\label{fig:typing}
\end{figure}

