\section{Typing}\label{sec:typing}

A Tiny Grace program $(\*\methh{\H_i}{e_i}*, e)$ is well-formed if we can
produce a type $\tau$ for $e$ in the context of its object body.  This can be
expressed with the typing rules in Section~\ref{sec:typing-rules} as:

\begin{equation}
  \tag{\textsc{T-Prg}}\label{eq:type-prg}
  \AxiomC{$\ctx{\varnothing}{\object{\*\methh{\H_i}{e_i}*}}{\type{\*\H_i*}}$}
  \AxiomC{$\ctx{\self : \type{\*\H_i*}}{e}{\tau}$}
  \BinaryInfC{$\pjdg{\*\methh{\H_i}{e_i}*}{e}$}
  \DisplayProof
\end{equation}

\subsection{Subtyping}\label{sec:subtyping}

The definition of the subtyping operator $\sub$ is given in
Figure~\ref{fig:subtyping}.  A type $\tau$ is a subtype of any type
$\tau^\prime$ if, for every method in $\tau^\prime$, there is a method with the
same name in $\tau$ with contravariant parameter types and a covariant return
type.  Subtyping over a sequence is written as $\*\tau^\prime_i <: \tau_i*$.

\begin{figure}[h]
  \centering

  \begin{equation}
    \tag{\textsc{T-Sub}}\label{eq:type-sub}
    \AxiomC{$\overline{\method{m_i}{\*x_{ik} : \upsilon_{ik}*}{\sigma_i}
      {e_i} \in \*\H_j*}$}
    \AxiomC{$\*\upsilon_{ik}^\prime \sub \upsilon_{ik}*$}
    \AxiomC{$\*\sigma_i \sub \sigma_i^\prime*$}
    \TrinaryInfC{$\type{\*\H_j*} \sub
      \type{\overline{\method{m_i}{\*x_{ik}^\prime : \upsilon_{ik}^\prime*}
        {\sigma_i^\prime}{e_i^\prime}}}$}
    \DisplayProof
  \end{equation}

  \caption{Tiny Grace subtyping}\label{fig:subtyping}
\end{figure}

\subsection{Typing Rules}\label{sec:typing-rules}

The typing rules for expressions and method declarations is given in
Figure~\ref{fig:typing}.  An environment $\G$ is a finite mapping from variable
names to types, written $\*\xs_i : \tau_i*$.  The empty environment is written
$\varnothing$.  Environments can be concatenated together, typically written
$\G,\*\xs_i : \tau_i*$, and variables that exist in both environments use the
mapping in the latter environment.  The typing judgement for expressions has the
form $\gctx{e}{\tau}$, meaning that in the environment $\G$, $e$ has the type
$\tau$.  The typing judgement for methods has the form $\gmctx{\M}$, meaning
that in the environment $\G$, the method $\M$ is well-formed. The typing
judgement for programs has the form $\pjdg{\*\M*}{e}$, meaning that the
program $(\*\M*, e)$ is well-formed.

\begin{figure}[h]
  \centering

  \newcommand{\name}[1]{\tag{\textsc{T-#1}}}

  \begin{equation}
    \name{Var}\label{eq:type-var}
    \AxiomC{$\gctx{\xs}{\G(\xs)}$}
    \DisplayProof
  \end{equation}

  \begin{equation}
    \name{Req}\label{eq:type-req}
    \AxiomC{$\gctx{e}{\type{\*\H_i*}}$}
    \AxiomC{$m(\*x_i : \upsilon_i*) \to \sigma \in \*\H_i*$}
    \AxiomC{$\gctx{\*e_i}{\upsilon^\prime_i}*$}
    \AxiomC{$\*\upsilon^\prime_i \sub \upsilon_i*$}
    \QuaternaryInfC{$\gctx{e.m(\*e_i*)}{\sigma}$}
    \DisplayProof
  \end{equation}

  \begin{equation}
    \name{Obj}\label{eq:type-obj}
    \AxiomC{$\overline{m_i \notin \*\H_i*\setminus\H_i}$}
    \AxiomC{$\G, \self : \type{\*\H_i*} \vdash
      \*\methh{\H_i}{e_i}~\mathrm{OK}*$}
    \BinaryInfC{$\gctx{\object{\*\methh{\H_i}{e_i}*}}{\type{\*\H_i*}}$}
    \DisplayProof
  \end{equation}

  \begin{equation}
    \name{Meth}\label{eq:type-meth}
    \AxiomC{$\*x_i \notin \G*$}
    \AxiomC{$\ctx{\G, \*x_i : \upsilon_i*}{e}{\sigma^\prime}$}
    \AxiomC{$\sigma^\prime \sub \sigma$}
    \TrinaryInfC{$\gmctx{\method{m}{\*x_i : \upsilon_i*}{\sigma}{e}}$}
    \DisplayProof
  \end{equation}

  \caption{Tiny Grace expression and method typing rules}\label{fig:typing}
\end{figure}

\subsection{Example}\label{sec:typing-example}

\begin{scope}

  \newcommand{\name}[1]{\tag{\textsc{Ex-#1}}}
  \def\Gp{\G, \self : \idt}
  \def\wrh{\mlst{wrap}({\param{x}{}}) \to {\vlt}}
  \def\wrt{\overset{w}\tau}
  \def\wrm{\methh{\wrh}{\vob}}
  \def\vlh{\head{value}{}{}}
  \def\vlt{\overset{v}\tau}
  \def\vlm{\methh{\vlh}{\mlst{x}}}
  \def\idt{\overset{id}\tau}
  \def\idm{\method{\mlst{id}}{\param{y}{}}{\type{}}{\mlst{y}}}
  \def\vob{\overset{v}\O}
  \def\aob{\overset{id}\O}
  \def\e{\self.\mlst{wrap}(\aob)}

  Continuing the example from the previous section, we can ensure that the
  example program is well-formed.  For the sake of space, we define some initial
  bindings:

  \begin{lstlisting}
@$\vlt =$@ { value() -> {} }
@$\wrt =$@ { wrap(x : {}) -> @$\vlt$@ }
@$\idt =$@ { id(y : {}) -> {} }
@$\vob =$@ object { method value() -> {} { x } }
@$\aob =$@ object { method id(y : {}) -> {} { y } }
  \end{lstlisting}

  \noindent Working backwards, we can begin by applying Rule~\ref{eq:type-prg}
  to the example program.
%
  \begin{equation}
    \name{Prg}\label{eq:typex-prg}
    \AxiomC{\ref{eq:typex-mod}}
    \AxiomC{\ref{eq:typex-req}}
    \BinaryInfC{$\pjdg{\wrm}{\e}$}
    \DisplayProof
  \end{equation}

  \noindent First we need to type the object body with Rule~\ref{eq:type-obj}:
%
  \begin{equation}
    \name{Mod}\label{eq:typex-mod}
    \AxiomC{$\mlst{wrap} \notin \varnothing$}
    \AxiomC{\ref{eq:typex-wrap}}
    \BinaryInfC{$\ctx{\varnothing}{\object{\wrm}}{\wrt}$}
    \DisplayProof
  \end{equation}

  \noindent This requires that we ensure the \lstinline{wrap} method is
  well-formed in the environment holding the object type as $\self$, defined as
  $\G = \self : \wrt$, using Rule~\ref{eq:type-meth}:
%
  \begin{equation}
    \name{Wrap}\label{eq:typex-wrap}
    \AxiomC{$x \notin \G$}
    \AxiomC{$\mlst{value} \notin \varnothing$}
    \AxiomC{$\ctx{\G, \param{x}{}}{\mlst{x}}{(\G, \param{x}{})(\mlst{x})}$}
    \UnaryInfC{$\ctx{\G, \param{x}{}}{\mlst{x}}{\type{}}$}
    \AxiomC{$\type{} \sub \type{}$}
    \BinaryInfC{$\mctx{\G, \param{x}{}}{\vlm}$}
    \BinaryInfC{$\ctx{\G, \param{x}{}}{\vob}{\vlt}$}
    \AxiomC{$\vlt \sub \vlt$}
    \TrinaryInfC{$\gmctx{\wrm}$}
    \DisplayProof
  \end{equation}

  \noindent The method body now has the type $\wrt$, so we apply
  Rule~\ref{eq:type-req} to the expression with the environment $\G = \self :
  \wrt$:
%
  \begin{equation}
    \name{Req}\label{eq:typex-req}
    \AxiomC{$\gctx{\self}{\wrt}$}
    \AxiomC{$\wrh \in \wrt$}
    \AxiomC{\ref{eq:typex-arg}}
    \AxiomC{$\idt \sub \type{}$}
    \QuaternaryInfC{$\gctx{\self.\mlst{wrap}(\aob)}{\vlt}$}
    \DisplayProof
  \end{equation}

  \noindent Finally, we have to ensure that the argument of the request to the
  \lstinline{wrap} method does in fact have the type $\idt$, by once again apply
  Rule~\ref{eq:type-obj}:
%
  \begin{equation}
    \name{Arg}\label{eq:typex-arg}
    \AxiomC{$y \notin \Gp$}
    \AxiomC{$\ctx{\Gp, y : \type{}}{y}{(\Gp, y : \type{})(y)}$}
    \UnaryInfC{$\ctx{\Gp, y : \type{}}{y}{\type{}}$}
    \AxiomC{$\type{} \sub \type{}$}
    \TrinaryInfC{$\mctx{\Gp}{\idm}$}
    \UnaryInfC{$\gctx{\aob}{\idt}$}
    \DisplayProof
  \end{equation}

  \noindent This completes the branches of the example~\ref{eq:typex-prg},
  confirming that the example program is well-formed.

\end{scope}

