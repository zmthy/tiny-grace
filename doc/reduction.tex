\section{Reduction}\label{sec:reduction}

Given a Tiny Grace program $(\*\M*, e)$, we can `run' $e$ in the scope of
$\*\M*$ by applying substitution from Section~\ref{sec:substitution} and
big-step reduction from Section~\ref{sec:reduction-semantics} to
$[\object{\*\M*}/\self]e \stot \O$, producing an outcome in the form of the
object literal $\O$.

\subsection{Substitution}\label{sec:substitution}

The substitution used in Tiny Grace reduction is non-trivially different from
standard substitution, so it is redefined in Figure~\ref{fig:substitution}.
Substitution in Tiny Grace is written $[\O/\xs]e$.  The important distinction is
Rule~\ref{eq:sub-self}, which prevents $\self$ from changing for inner objects.

\begin{figure}[h]
  \centering
  \newcommand{\name}[1]{\tag{\textsc{S-#1}}}
  \addtolength{\parskip}{-1em}

  \begin{equation}
    \name{Var}\label{eq:sub-var}
    \AxiomC{$[\O/\xs]\xs \bto \O$}
    \DisplayProof
  \end{equation}
%
  \begin{equation}
    \name{None}\label{eq:sub-none}
    \AxiomC{$[\O/\xs]y \bto y \quad \mathbf{if}~\xs \neq y$}
    \DisplayProof
  \end{equation}
  \vspace{-.8em}
  \begin{equation}
    \name{Self}\label{eq:sub-self}
    \AxiomC{$[\O/\self]\object{\*\M_i*} \bto \object{\*\M_i*}$}
    \DisplayProof
  \end{equation}

  \begin{equation}
    \name{Obj}\label{eq:sub-obj}
    \AxiomC{$\*[\O/x]e_i \bto e_i^\prime*$}
    \UnaryInfC{$[O/x]\object{\*\methh{\H_i}{e_i}*} \bto
      \object{\*\methh{\H_i}{e_i^\prime}*}$}
    \DisplayProof
  \end{equation}

  \begin{equation}
    \name{Req}\label{eq:sub-req}
    \AxiomC{$[\O/\xs]e \bto e^\prime$}
    \AxiomC{$\*[\O/\xs]e_i \bto e_i^\prime*$}
    \BinaryInfC{$[\O/\xs]e.m(\*e_i*) \bto e^\prime.m(\*e_i^\prime*)$}
    \DisplayProof
  \end{equation}

  \caption{Big-step semantics of Tiny Grace substitution}\label{fig:substitution}
\end{figure}

\subsection{Reduction Semantics}\label{sec:reduction-semantics}

The rules for the reduction relation $\sto$ are given in
Figure~\ref{fig:reduction}.  The relation is written $e \sto e^\prime$, meaning
that the expression $e$ reduces to $e^\prime$ in a single small-step reduction
step.  We write $\stot$ for the reflexive transitive closure of $\sto$.  The
reduction $e \stot \O$ is the big-step form of the reduction semantics.

\begin{figure}[h]
  \centering

  \begin{equation}
    \tag{\textsc{R-Rec}}\label{eq:red-rec}
    \AxiomC{$e \sto e^\prime$}
    \UnaryInfC{$e.m(\*e_i*) \sto e^\prime.m(\*e_i*)$}
    \DisplayProof
  \end{equation}

  \begin{equation}
    \tag{\textsc{R-Arg}}\label{eq:red-arg}
    \AxiomC{$e_i \sto e_i^\prime$}
    \UnaryInfC{$\O.m(\*\O_i*, e_i, \*e_i*) \sto
      \O.m(\*\O_i*, e_i^\prime, \*e_i*)$}
    \DisplayProof
  \end{equation}

  \begin{equation}
    \tag{\textsc{R-Req}}\label{eq:red-req}
    \AxiomC{$\method{m}{\*x_i : \upsilon_i*}{\sigma}{e} \in \*\M*$}
    \UnaryInfC{$\object{\*\M*}.m(\*\O_i*) \sto [\*\O_i/x_i*,\ \object{\*\M*}/\self]e$}
    \DisplayProof
  \end{equation}

  \caption{Small-step semantics of Tiny Grace reduction}\label{fig:reduction}
\end{figure}

\subsection{Example}

\begin{scope}
  \def\ov{\overset{v}{\O}}
  \def\oid{\overset{id}{\O}}

  \def\x{\mlst{x}}
  \def\y{\mlst{y}}

  \def\vtype{\type{\head{value}{}{}}}
  \def\wrap{\method{\mlst{wrap}}{\x : \type{}}{\vtype}{\ov}}
  \def\mid{\method{\mlst{id}}{\y : \type{}}{\type{}}{\x}}
  \def\exp{\self.\mlst{wrap}\text{\po}\oid\text{\pc}}

  As the Section~\ref{sec:typing-example} demonstrated that the example program
  was well-formed, we can run the program. Beginning with the example program:

  \begin{lstlisting}
(method wrap(x : {}) -> { value() -> {} } {
  object { method value() -> {} { x } }
}, self.wrap(object { method id(y : {}) -> {} { y } }))
  \end{lstlisting}

  \noindent We run the program by substituting the surrounding methods in an
  object for $\self$.

  \begin{lstlisting}
[object{
  method wrap(x : {}) -> { value() -> {} } {
    object { method value() -> {} { x } }
  }
}/self]self.wrap(object { method id(y : {}) -> {} { y } })
  \end{lstlisting}

  \noindent This substitution uses Rule~\ref{eq:sub-var} to replace $\self$, as
  well as Rule~\ref{eq:sub-self} on the argument to the \lstinline{wraq} method,
  which has no effect and does not proceed further.

  \begin{lstlisting}
object{
  method wrap(x : {}) -> { value() -> {} } {
    object { method value() -> {} { x } }
  }
}.wrap(object { method id(y : {}) -> {} { y } })
  \end{lstlisting}

  \noindent Now a step of reduction takes place.  The top-level expression is a
  request, so we apply Rule~\ref{eq:red-req}, replacing the request with a
  substitution over the appropriate method's body.

  \begin{lstlisting}
[object { method id(y : {}) -> {} { y } }/x, object {
  method wrap(x : {}) -> { value() -> {} } {
    object { method value() -> {} { x } }
  }
}/self]object { method value() -> {} { x } }
  \end{lstlisting}

  \noindent The substitution of $\self$ applies Rule~\ref{eq:sub-self}, which
  again has no effect.

  \begin{lstlisting}
[object{ method id(y : {}) -> {} { y } }/x]object{ method value() -> {} { x } }
  \end{lstlisting}

  \noindent The remaining substitution applies Rule~\ref{eq:sub-obj}, which
  reaches into the object's method, replacing its body:

  \begin{lstlisting}
[object { method id(y : {}) -> {} { y } }/x]x
  \end{lstlisting}

  \noindent This is a simple application of Rule~\ref{eq:sub-var}.

  \begin{lstlisting}
object { method id(y : {)}) -> {} { y } }
  \end{lstlisting}

  \noindent Completing the application of Rule~\ref{eq:sub-obj} produces the
  following expression:

  \begin{lstlisting}
object {
  method value() -> {} {
    object { method id(y : {}) -> {} { y }}
  }
}
  \end{lstlisting}

  \noindent As this expression is an object literal, there are no more rules
  which can be applied and the execution is complete.
\end{scope}

