\section{Reduction}\label{sec:reduction}

Given a Tiny Grace program $(\ol{M}, e)$, we can `run' $e$ in the scope of
$\ol{M}$ by applying substitution from Section~\ref{sec:substitution} and
big-step reduction from Section~\ref{sec:reduction-semantics} to
$[\object{\ol{M}}/\self]e \stot O$, producing an outcome in the form of the
object literal $O$.

\subsection{Substitution}\label{sec:substitution}

The substitution used in Tiny Grace reduction is non-trivially different from
standard substitution, so it is redefined in Figure~\ref{fig:substitution}.
Substitution in Tiny Grace is written $[O/\xs]e$. The important distinction is
Rule~\ref{eq:sub-self}, which prevents $\self$ from changing for inner objects.

\begin{figure}[h]
  \centering
  \newcommand{\name}[1]{\tag{\textsc{S-#1}}}
  \addtolength{\parskip}{-1em}

  \begin{equation}
    \name{Var}\label{eq:sub-var}
    \AxiomC{$[O/\xs]\xs\bto O$}
    \DisplayProof
  \end{equation}
%
  \begin{equation}
    \name{None}\label{eq:sub-none}
    \AxiomC{$[O/\xs]y\bto y\quad\mathbf{if}~\xs\neq y$}
    \DisplayProof
  \end{equation}
  \vspace{-.8em}
  \begin{equation}
    \name{Self}\label{eq:sub-self}
    \AxiomC{$[O/\self]\object{\ol{M}}\bto\object{\ol{M}}$}
    \DisplayProof
  \end{equation}

  \begin{equation}
    \name{Obj}\label{eq:sub-obj}
    \AxiomC{$\ol{[O/x]e\bto e\′}$}
    \UnaryInfC{$[O/x]\object{\ol{\methh{H}{e}}} \bto
      \object{\ol{\methh{H}{e\′}}}$}
    \DisplayProof
  \end{equation}

  \begin{equation}
    \name{Req}\label{eq:sub-req}
    \AxiomC{$[O/\xs]e\bto e\′$}
    \AxiomC{$\ol{[O/\xs]e_a\bto e_a\′}$}
    \BinaryInfC{$[O/\xs]e.m(\ol{e_a})\bto e\′.m(\ol{e_a\′})$}
    \DisplayProof
  \end{equation}

  \caption{Big-step semantics of Tiny Grace substitution}\label{fig:substitution}
\end{figure}

\subsection{Reduction Semantics}\label{sec:reduction-semantics}

The rules for the reduction relation $\sto$ are given in
Figure~\ref{fig:reduction}. The relation is written $e \sto e\′$, meaning that
the expression $e$ reduces to $e\′$ in a single small-step reduction step. We
write $\stot$ for the reflexive transitive closure of $\sto$. The reduction $e
\stot O$ is the big-step form of the reduction semantics.

\begin{figure}[h]
  \centering

  \begin{equation}
    \tag{\textsc{R-Rec}}\label{eq:red-rec}
    \AxiomC{$e\sto e\′$}
    \UnaryInfC{$e.m(\ol{e_a})\sto e\′.m(\ol{e_a})$}
    \DisplayProof
  \end{equation}

  \begin{equation}
    \tag{\textsc{R-Arg}}\label{eq:red-arg}
    \AxiomC{$e\sto e\′$}
    \UnaryInfC{$O.m(\ol{O}, e, \ol{e_a}) \sto
      O.m(\ol{O}, e\′, \ol{e_a})$}
    \DisplayProof
  \end{equation}

  \begin{equation}
    \tag{\textsc{R-Req}}\label{eq:red-req}
    \AxiomC{$\method{m}{\ol{x:\tau_p}}{\tau}{e}\in\ol{M}$}
    \UnaryInfC{$\object{\ol{M}}.m(\ol{O})\sto[\ol{O/x},\object{\ol{M}}/\self]e$}
    \DisplayProof
  \end{equation}

  \caption{Small-step semantics of Tiny Grace reduction}\label{fig:reduction}
\end{figure}

