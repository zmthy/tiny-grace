\section{Reduction}\label{sec:reduction}

The rules for the reduction relation $\ssr$ are given in
Figure~\ref{fig:reduction}. The relation is written $\sto{e}{e\′}$, meaning that
the expression $e$ reduces to $e\′$ in a single small-step reduction step. We
write $\ssrt$ for the reflexive and transitive closure of $\ssr$. Given a Tiny
Grace program $(\ol{M},e)$, we can `run' $e$ in the scope of $\ol{M}$ by
applying the substitution $[\object{\ol{M}}/\self]e$, and then applying
$\ssrt$, resulting in either an object literal $O$ or divergence.

\begin{figure}
  \centering
  \newcommand{\name}[1]{\tag{\textsc{R-#1}}}

  \begin{equation*}
    \name{Recv}\label{eq:red-recv}
    \AxiomC{$\sto{e}{e\′}$}
    \UnaryInfC{$\sto{e.m(\ol{e_a})}{e\′.m(\ol{e_a})}$}
    \DisplayProof
  \end{equation*}

  \begin{equation*}
    \name{Prm}\label{eq:red-prm}
    \AxiomC{$\sto{e}{e\′}$}
    \UnaryInfC{$\sto{\recO.m(\ol{\prmO},e,\ol{\prme})}
      {\recO.m(\ol{\prmO},e\′,\ol{\prme})}$}
    \DisplayProof
  \end{equation*}

  \begin{equation*}
    \name{Req}\label{eq:red-req}
    \AxiomC{$\method{m}{\ol{x:\prmt}}{\rett}{\rete}\in\recO$}
    \UnaryInfC{$\sto{\recO.m(\ol{\prmO})}
      {[\recO/\self,\ol{\prmO/x}]\rete}$}
    \DisplayProof
  \end{equation*}

  \caption{Small-step semantics of Tiny Grace reduction}\label{fig:reduction}
\end{figure}

