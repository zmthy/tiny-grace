\documentclass[a4paper, 11pt]{article}

\usepackage{fullpage}
\usepackage[T1]{fontenc}

\usepackage{amsmath}
\usepackage{mathtools}
\usepackage{bussproofs}

\usepackage{syntax}


% Document information

\def\version{0.1.0}

\def\thetitle{Tiny Grace Formal Specification}
\def\theauthor{Timothy Jones}

\usepackage[
pdftitle={\thetitle},
pdfauthor={\theauthor},
colorlinks,
urlcolor=blue,
linkcolor=blue,
citecolor=blue
]{hyperref}


% Math definitions

\def\M{\mathcal{M}}
\def\B{\mathcal{B}}

\def\bo{{\sffamily\{}}
\def\bc{{\sffamily\}}}


\title{\thetitle\\\normalsize\vspace{0.5em}Version \version\vspace{-0.5em}}
\author{\theauthor}

\bibliographystyle{acm}

\begin{document}
\maketitle\vspace{-2em}

\section{Introduction}

Tiny Grace is a tiny subset of the Grace programming language, in the spirit of
Featherweight Java \cite{fj}.  It currently aims to resolve the following
issues:

\begin{itemize}

    \item Simple reduction semantics

    \item Static typing

\end{itemize}

\noindent Future versions will aim to extend the formalism to address:

\begin{itemize}

    \item Inheritance semantics and typing

    \item Dialect semantics

    \item Statements and early returns

    \item Block semantics and typing

    \item Dynamic typing and runtime semantics

\end{itemize}

\noindent Tiny Grace intentionally avoids the following issues:

\begin{itemize}

    \item Concrete syntax and sugar, such as mixfix declarations and classes

    \item Variables and the heap

    \item Generic and inferred types

    \item Annotations and reflection

\end{itemize}

\noindent The current version of Tiny Grace is much simpler than even
Featherweight Java, as it does not feature constructors, fields or casts.
Instead of these features, it must implement scope, as objects can be nested.


\section{Syntax}

\def\*#1*{$\overline{#1}$}

The abstract syntax for Tiny Grace can be found in
figure~\ref{fig:abstract-syntax}.  A Tiny Grace program consists of an object
body at the top level, which is a set of method declarations.  The metavariable
$\B$ ranges over object bodies; $x$ over variable names; $\tau$ and $\sigma$ over
types; $\M$ over method headers, $m$ over method names, and $e$ over expressions.

We write \*e* to indicate a possibly empty sequence of comma-separated
expressions $e_1, \dots, e_n$, as well as for method header parameters \*x :
\tau*.  We also write \*\M* to indicate a possibly empty set of non-separated
method headers $\M_1 \dots \M_n$, as well as for method declarations in an
object body.

\begin{figure}

\def\*#1*{$\overline{\text{#1}}$}
\def\<#1>{$#1$}

\grammarindent3.4em
\renewcommand{\grammarlabel}[2]{$#1$\hfill#2}
\renewcommand{\syntleft}{\itshape}
\renewcommand{\syntright}{}
\renewcommand{\ulitleft}{\sffamily\bfseries}
\renewcommand{\litleft}{\sffamily}
\renewcommand{\litright}{}

\begin{grammar}
<\B> ::= \*"method" $\M$ \bo~\<e> \bc*

<\M> ::= <m>`('\*\<x> $:$ $\tau$*`)' $\to$ $\tau$

<\tau> ::= \bo~\*$\M$* \bc

<e> ::= <x> | <e>`.'<m>(\*\<e>*) | "object" \bo~$\B$ \bc
\end{grammar}

\label{fig:abstract-syntax}

\caption{Tiny Grace abstract syntax}

\end{figure}


\section{Reduction}

Given a Tiny Grace program, we can reduce expressions as though they appeared at
the bottom of the outer object body.

\begin{figure}[h]
\centering

\def\*#1*{\overline{#1}}

\AxiomC{$e \to e^\prime$}
\RightLabel{(Receiver)}
\UnaryInfC{$e.m(\*e_i*) \to e^\prime.m(\*e_i*)$}
\DisplayProof

\vspace{1em}

\AxiomC{$e_i \to e_i^\prime$}
\RightLabel{(Argument)}
\UnaryInfC{$e.m(\dots, e_i, \dots) \to e.m(\dots, e_i^\prime, \dots)$}
\DisplayProof

\label{fig:reduction}
\caption{Tiny Grace reduction rules}

\end{figure}


\section{Typing Rules}

\begin{figure}
\centering

\def\sub{<\kern-1pt:}
\def\*#1*{\overline{#1}}

\AxiomC{$\tau \sub \tau^\prime$}
\AxiomC{$\tau^\prime \sub \sigma$}
\BinaryInfC{$\tau \sub \sigma$}
\DisplayProof
\hspace{1em}
\AxiomC{$\*\M* \subseteq \*\M^\prime*$}
\UnaryInfC{\bo~$\*\M^\prime*$ \bc~$\sub$ \bo~$\*\M*$ \bc}
\DisplayProof

\label{fig:subtyping}
\caption{Tiny Grace subtyping rules}

\end{figure}


\bibliography{references}

\end{document}

